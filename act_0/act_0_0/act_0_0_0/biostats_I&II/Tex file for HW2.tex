\documentclass[12pt]{article}
\usepackage{geometry,amsmath,amssymb, graphicx, natbib, float, enumerate}
\geometry{margin=1in}
\renewcommand{\familydefault}{cmss}
\restylefloat{table}
\restylefloat{figure}

\newcommand{\code}[1]{\texttt{#1}}
\newcommand{\Var}{\mathrm{Var}}
\newcommand{\logit}{\mathrm{logit}}

\begin{document}
\noindent
{\bf BST 140.651 \\ Problem Set 2} \\

\begin{enumerate}[Problem 1.]
\item  Using the rules of expectations prove that $\Var(X) = E[X^2] - E[X]^2$
  where $\Var(X) = E[(X - \mu)^2]$.
\item Let $g(x) = \pi_1 f_1(x) + \pi_2f_2(x)+\pi_3f_3(x)$ where $f_1$ and $f_2$
  are densities with associated means and variances $\mu_1$,
  $\sigma^2_1$, $\mu_2$, $\sigma^2_2$, $\mu_3$, $\sigma^2_3$, respectively. Here $\pi_1,\pi_2,\pi_3\geq 0$ and $\sum_{i=1}^3\pi_i=1$. Show that $g$ is a valid density.
  What is it's associated mean
  and variance?
\item  Suppose that a density is of the form $(k + 1)x^k$ for some constant $k > 1$ and $0 < x < 1$.
  \begin{enumerate}[a.]
  \item What is the mean of this distribution?
  \item What is the variance?
  \end{enumerate}
\item Suppose that the time in days until hospital discharge for a certain patient population
follows a density $f(x) = \frac{1}{3.3}\exp(-x/3.3)$ for $x > 0$.
\begin{enumerate}[a.]
\item Find the mean and variance of this distribution.
\item The general form of this density (the exponential density) is $f(x) = \frac{1}{\beta}\exp(-x/\beta)$ for $x > 0$ for a fixed value of $\beta$. Calculate the mean and variance of this density.
\item Plot the exponential pdf for $\beta=0.1, 1, 10$.
\end{enumerate}
\item The Gamma density is given by
    $$
    \frac{1}{\beta^\alpha\Gamma(\alpha)} x^{\alpha - 1} \exp(-x/\beta) ~~ \mbox{for} ~~ x > 0
    $$
for fixed values of $\alpha$ and $\beta$.
\begin{enumerate}[a.]
\item Derive the mean and variance of the gamma density. You can assume the fact (proved in HW 1) that the density integrates to 1 for any $\alpha > 0$ and $\beta > 0$.
\item The Chi-squared density is the special case of the Gamma density where $\beta = 2$ and
$\alpha = p / 2$ for some fixed value of $p$ (called the ``degrees of freedon''). Calculate the
mean and variance of the Chi-squared density.
\end{enumerate}
\item The Beta density is given by
$$
\frac{1}{B(\alpha, \beta)} x^{\alpha - 1}(1 - x)^{\beta - 1} ~~\mbox{for}~~ 0 < x< 1
$$
and
$
B(\alpha, \beta) = \Gamma(\alpha)\Gamma(\beta)/\Gamma(\alpha + \beta).
$
\begin{enumerate}[a.]
\item Derive the mean of the beta density. Note the following is useful for simplifying results: $\Gamma(c + 1) = c\Gamma(c)$ for $c > 0$.
\item Derive the variance of the beta density.
\end{enumerate}
\item The Poisson mass function is given by
$$
P(X = x) = \frac{e^{-\lambda} \lambda^x}{x!} ~~ \mbox{for} ~~ x = 0, 1, 2, 3, \ldots
$$
\begin{enumerate}[a.]
\item Derive the mean of this mass function.
\item Derive the variance of this mass function. Hint, consider $E[X(X - 1)]$.
\end{enumerate}
\item Suppose that, for a randomly drawn subject from a particular
  population, the proportion of a their skin that is covered in
  freckles follows a uniform density (constant between $0$ and $1$).
  \begin{enumerate}[a.]
  \item What is the expected percentage of a (randomly selected) person's body that is covered in freckles? (Show your work.)
  \item What is the variance? (Show your work.)
  \end{enumerate}
\item  You have an MP3 player with a total of $1000$ songs stored on it. Suppose that songs are played
  randomly {\em with replacement}. Let $X$ be the number of songs played
  until you hear a repeated song.
  \begin{enumerate}[a.]
  \item What values can $X$ take, and with what probabilities?
  \item What is the expected value for $X$?
  \item What is the variance for $X$?
  \end{enumerate}
\item When at the free-throw line for two shots, a basketball player makes at least one free throw
  90\% of the time. 80\% of the time, the player makes the first shot, while 70\% of the time
  she makes both shots.
\begin{enumerate}[a.]
\item Does it appear that the player's second shot success is independent of the first?
\item What is the conditional probability that the player makes the second shot given that she
  made the first? What would it be if she missed the first?
\end{enumerate}
\item  Assume that an act of intercourse between an HIV infected person
  and a non-infected person results in a $1/500$ probability of
  spreading the infection. How many acts of intercourse would an
  uninfected person have to have with an infected persons to have a
  10\% probability of obtaining an infection?  State the assumptions
  of your calculations.
\item You meet a person at the bus stop and strike up a conversation. In the conversation, it is revealed that the person is a parent of two children and that one of the two children is a girl. However, you do not know the gender of the other child, nor whether the daughter
she mentioned is the older or younger sibling.
\begin{enumerate}[a.]
\item What is the probability that the other sibling is a girl? What assumptions are you making to
perform this calculation?
\item Later in the conversation, it becomes apparent that she was discussing the older sibling.
Does this change your probability that the other sibling is a girl?
\end{enumerate}
\item A particularly sadistic warden has three prisoners, A, B and C. He tells prisoner C
that the sentences are such that two prisoners will be executed and let one free, though he
will not say who has what sentence. Prisoner C convinces the warden to tell
him the identity of one of the prisoners to be executed. The warden has the following strategy, which prisoner C is aware of.  If C is sentenced to be let free, the warden flips a coin to pick between A and B and tells prisoner C that person's sentence. If C is sentenced to be executed he gives the identity of whichever of A or B is also sentenced to be executed.
\begin{enumerate}[a.]
\item Does this new information about one of the other prisoners give prisoner C any
more information about his sentence?
\item The warden offers to let prisoner C switch sentences with the other prisoner whose sentence
he has not identified. Should he switch?
\end{enumerate}
\item The Chinese Mini-Mental Status Test (CMMS) is a test consisting
  of 114 items intended to identify people with Alzheimer's disease
  (AD) and dementia among people in China. An extensive clinical
  evaluation was performed of this instrument, whereby participants
  were interviewed by psychiatrists and nurses and a definitive
  (clinical) diagnosis of AD was made. The table below show the counts
  obtained on the subgroup of people with at least some formal education.
  Suppose a cutoff value of $\leq 20$ on the test is used to identify
  people with AD.
  \begin{center}
    \begin{tabular}{lccc}
        && \multicolumn{2}{c}{Clinical diagnosis of AD} \\ \cline{3-4}
     CMMS score && No & Yes \\ \hline
     0-5   && 0  & 2 \\
     6-10  && 0  & 1 \\
     11-15 && 3  & 4 \\
     16-20 && 9  & 5 \\
     21-25 && 16 & 3 \\
     26-30 && 18 & 1 \\ \hline
    \end{tabular}
  \end{center}
  \begin{enumerate}[a.]
  \item What is the sensitivity and specificity of the CMMS test using the 20 cutoff?
  \item Create a plot of the sensitivity by (1 - specificity), which is the true positive
    rate versus the false positive rate for all of the cut-offs between 0 and 30. This is
    called an ROC curve.
  \item Graph the positive predictive value as a function of the prevalence of AD.
    Do the same for the negative predictive value.
  \end{enumerate}
\item  A web site (www.medicine.ox.ac.uk/bandolier/band64/b64-7.html) for home pregnancy tests cites the following:
\begin{quote}
  When the subjects using the test were women who collected and tested
  their own samples, the overall sensitivity was 75\%. Specificity was
  also low, in the range 52\% to 75\%.
\end{quote}
\begin{enumerate}[a.]
\item Interpret a positive and negative test result using diagnostic
  likelihood ratios using both extremes of the specificity.
\item A woman taking a home pregnancy test has a positive test. Draw a
  graph of the positive predictive value by the prior probability
  (prevalence) that the woman is pregnant. Assume the specificity is 63.5\%
\item Repeat the previous question for a negative test and the negative
  predictive value.
\end{enumerate}
 \item Given below are the sexes of the children of 7,745 families of 4 children
    recorded in the archives of the Genealogical Society of the Church of Jesus Christ
    of Latter Day Saints in Salt Lake City, Utah. $M$ indicates a male child and
    $F$ indicates a female child.
    \begin{center}
    \begin{tabular}{llll}
      Sequence & Freq & Sequence & Freq \\ \hline
      MMMM & 537 & MFFM & 526 \\
      MMMF & 549 & FMFM & 498 \\
      MMFM & 514 & FFMM & 490 \\
      MFMM & 523 & MFFF & 429 \\
      FMMM & 467 & FMFF & 451 \\
      MMFF & 497 & FFMF & 456 \\
      MFMF & 486 & FFFM & 441 \\
      FMMF & 473 & FFFF & 408 \\ \hline
    \end{tabular}
    \end{center}
    \begin{enumerate}[a.]
    \item Estimate the probability distribution of the number of male children, say $X$,
      in these families using the data below by calculating proportions.
    \item Find the expected value of $X$.
    \item Find the variance of $X$.
    \item Find the probability distribution of $\hat p$, where $\hat p$ is the
      proportion of  children in each family who are male. Find the expected value of $\hat p$
      and the variance of $\hat p$
    \end{enumerate}
  \item  Quality control experts estimate that the time (in years) until a specific electronic
    part from an assembly line fails follows (a specific instance of) the {\bf Pareto} density
    $$
    \frac{3}{x^4} ~~~~~~~\mbox{for}~ 1 < x < \infty.
    $$
    \begin{enumerate}[a.]
    \item What is the average failure time for components from this density? (Show your work.)
    \item What is the variance? (Show your work.)
    \item The general form of the Pareto density is given by $\frac{\beta \alpha^\beta}{x^{\beta + 1}}$
for $0 < \alpha < x$ and $\beta > 0$ (for fixed $\alpha$ and $\beta$). Calculate the mean and variance of the general Pareto density.
    \end{enumerate}
  \item You are playing a game with a friend where you flip a coin and if it comes up heads you
    give him a dollar and if it comes up tails she gives you a dollar. You play the game ten times.
    \begin{enumerate}[a.]
    \item What is the expected total earnings for you? (Show your work; state your assumptions.)
    \item What is the variance of your total earnings? (Show your work; state your assumptions.)
    \item Suppose that the coin is biased and you have a $.4$ chance of winning for each flip.
      repeat the calculations in parts $a$ and $b$
    \end{enumerate}
  \item Note that the code
\begin{verbatim}
temp <- matrix(sample(1 : 6, 1000 * 10, replace = TRUE), 1000)
xBar <- apply(temp, 1, mean)
\end{verbatim}
    In R produces $1,000$ averages of $10$ die rolls. That is, it's
    like taking ten dice, rolling them, averaging the results and
    repeating this $1,000$ times.
    \begin{enumerate}[a.]
    \item Do this in R. Plot histograms of the averages.
    \item Take the mean of \texttt{xBar}. What should this value be close to? (Explain your reasoning.)
    \item Take the standard deviation of \texttt{xBar}. What should this value be close to? (Explain your reasoning.)
    \end{enumerate}
  \item Note that the code
\begin{verbatim}
xBar <- apply(matrix(runif(1000 * 10), 1000), 1, mean)
\end{verbatim}
    produces $1,000$ averages of $10$ uniforms.
    \begin{enumerate}[a.]
    \item Do this in R. Plot histograms of the averages.
    \item Take the mean of \texttt{xBar}. What should this value be close to? (Explain your reasoning.)
    \item Take the standard deviation of \texttt{xBar}. What should this value be close to? (Explain your reasoning.)
    \end{enumerate}
\end{enumerate}

\end{document}
