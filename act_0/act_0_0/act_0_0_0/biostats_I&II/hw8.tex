\documentclass[12pt]{article}
\usepackage{geometry,amsmath,amssymb, graphicx, natbib, float, enumerate}
\geometry{margin=1in}
\renewcommand{\familydefault}{cmss}
\restylefloat{table}
\restylefloat{figure}

\newcommand{\code}[1]{\texttt{#1}}
\newcommand{\Var}{\mathrm{Var}}
\newcommand{\logit}{\mathrm{logit}}
\newcommand{\RQ}{[{\bf REQUIRED}]~}

\begin{document}
\noindent
{\bf BST 140.652 \\ Problem Set 8 } \\

\begin{enumerate}[Problem 1.]
\item Obstetric records of a group of children who died suddenly and unexpectedly,
S.U.D., were compared with those of a group of live control children.  Observations
on the duration of the 2nd state of labor were as follows:
\begin{center}
Time in minutes:
\ttfamily
\begin{tabular}{lccccccccccc}
\\ \hline
S.U.D. & 60, & 25, & 6, & 8, & $<$5, & 10, & 25, & 15, & 10, \vspace{+0.25in} \\
Controls & 13, & 20, & 15, & 7, & 75, & 120$^\star$, & 10, & 100, & 9, & 25, & 30
\\ \hline
\end{tabular}
\normalfont
$^\star$ Terminated by surgical intervention
\end{center}
Compare the median duration of labor in the two groups and test the
significance of the difference using the Wilcoxon test.
\item Independent random samples of size 4 and 2 are taken from 2
  continuous populations. Enumerate all possible rank collections for
  the smaller sample.  Obtain the null distribution of the
  Mann-Whitney (Wilcoxon-Rank Sum) test statistic.
\item A large survey of over 100,000 births in South Wales during the period
1956-1962 gave an incidence rate for spina bifida of 4.12 per
1,000 births.  In a random sample of 1000 births, compute the
probability of observing (i) no cases, (ii) one case, (iii) two
cases.  Using the following two approaches
\begin{enumerate}[a.]
\item The exact distribution based on the binomial distribution
\item Approximate probabilities based on a Poisson approximation to the binomial
\end{enumerate}
\item A cohort study was performed to evaluate the relation between
  oral-contraceptive use and breast cancer in the age group 40-44. At
  enrollment, women were classified as current user, past user or
  never user of oral contraceptives.
\begin{center}
\tabcolsep=0.25in
\begin{tabular}{lcc}
OC Use & Cancer Cases & Person-Years \vspace{+0.15in} \\ \hline
Current user & 13 & 4,761 \vspace{+0.15in} \\
Past user & 164 & 121,091 \vspace{+0.15in} \\
Never user & 113 & 98,091 \vspace{+0.15in} \\ \hline
\end{tabular}
\end{center}
\begin{enumerate}[a.]
\item Calculate the breast cancer incidence rates in the 3 groups
  along with 95\% confidence intervals. State your assumptions.
\item Is there a significant difference in incidence rates between
  current users and never users?  How about between past users and
  never users?  Perform hypothesis test, report a p-value and
  interpret your findings.
\item Calculate the ratio of incidence rates comparing current users
  versus never users along with a 95\% confidence interval for this
  ratio.
\end{enumerate}
\item Researchers comparing fMRI signals between a resting state and a active state
  in 10 different regions of the brain, found the following P-values
\begin{center}
  \begin{tabular}{lllllllllll}
Region  & 1 & 2 & 3 & 4 & 5 & 6 & 7 & 8 & 9 & 10 \\ \hline
P-value & .081 & .011 & .053 & .0140 & .016 & .045 &
          .046 & .050 & .003 & .053 \\ \hline
  \end{tabular}
\end{center}
\begin{enumerate}[a.]
\item Controlling the FWER of .05, which regions would be rejected? (Interpret your results.)
\item Controlling the FDR at .05, which regions would be rejected? (Interpret your results.)
\end{enumerate}
\item The following data show blood glucose levels in mg/kg in rabbits 
immediately before and two hours after the administration of an analgesic compound.
\begin{description}
\item Investigate the effect of analgesia on blood glucose level by applying
(1) the sign test, (2) the Wilcoxon signed rank test, (3) an appropriate 
parametric test.
\begin{center}
\begin{tabular}{ccc}
Rabbit number & Before analgesia & After analgesia \\ \hline
1 & 158 & 206 \\
2 & 119 & 134 \\
3 & 122 & 204 \\
4 & 89 & 105 \\
5 & 111 & 96 \\
6 & 135 & 171 \\
7 & 138 & 212 \\
8 & 122 & 134 \\
9 & 127 & 177 \\
10 & 127 & 136 \\
11 & 137 & 137 \\
12 & 120 & 117 \\
13 & 118 & 127 \\
14 & 126 & 140 \\
15 & 134 & 153 \\
16 & 134 & 147 \\
17 & 125 & 131 \\
18 & 124 & 131 \\ \hline
\end{tabular}
\end{center}
Interpret your results.  Comment on any assumptions the methods make.  Please
do these calculations first without a computer, then if you like, try with a
computer.  Describe the computer procedure you used and whether your results agree.
\end{description}
\item Medical records were reviewed to assess the frequency of medical
malpractice.  Two reviewers classified the same records.  We wish to study the
agreement between the reviewers. Perform an analysis to assess agreement between
the reviewers ($+$ means classified as malpractice).
\begin{center}
\tabcolsep= 0.25in
\begin{tabular}{llcc}
&& \multicolumn{2}{c}{Review A} \\
&& $+$ & $-$ \\
Reviewer B & $+$ & 35 & 13 \\
& $-$ & 21 & 249 
\end{tabular}
\end{center}
\vspace{+0.25in}
\item A paired comparison experiment will be performed with $n=15$ pairs of
subjects.
\begin{description}
\item (a) To test $H_0: P+ = .5 \; {\rm vs.} \; H_1: P+ < .5$ with the sign 
test, determine the rejection region so that $\alpha$ is closest $.10$.
\item (b) Using the binomial table, find the power of the test corresponding to
each of the alternatives $P+ = .05, .1, .2, .3, .4$.  Sketch the power curve
for all values of $P+ (0 < P+ < 1)$.
\end{description}
\item Consider the Wilcoxon-Signed Rank Test with $n = 4$.
\begin{description}
\item (a) List all possible associations of signs with the ranks
$1, 2, 3, 4$ and list corresponding values of $T^+$ and $T^-$.
\item (b) Obtain the probability distribution of $T^+$ and $T^-$ for $n=4$ using (a).
\item (c) From (b) calculate ${\rm E}(T^+), {\rm Var}(T^+)$.  Also compute
these from the formulas given for a general $n$.
\end{description}
\item A new drug is developed to relieve the ocular symptoms (redness, itching) 
of hay fever.  An experiment is performed on a group of 24 patients with hay fever.
In the experiment drug A is administered to a randomly selected eye and a placebo
is administered to the other eye and the change from baseline is noted for each
eye.  The data are given; $+$ represents more improvement in the drug-treated eye;
$-$ represents more improvement in the placebo-treated eye; $0$ represents 
equal improvement in both eyes.
\begin{center}
Comparison of drug A vs. placebo for the relief of redness
and itching in hay fever patients
\begin{tabular}{ccccccccc}
\\ \hline
Subject & Redness & Itching & Subject & Redness & Itching & Subject & Redness
& Itching \\ \hline
1 & $+$ & $+$ & 9  & $-$ & $+$ & 17 & $-$ & $+$ \\
2 & 0 & $+$   & 10 & $-$ & $-$ & 18 & $-$ & 0 \\
3 & $-$ & $-$ & 11 & $+$ & $+$ & 19 & 0   & 0 \\
4 & $+$ & $-$ & 12 & 0 & $+$ & 20   & $-$ & $+$ \\
5 & $+$ & $+$ & 13 & $+$ & $+$ & 21 & $-$ & 0 \\
6 & $+$ & $+$ & 14 & $-$ & 0 & 22 & $-$ & $+$ \\
7 & $+$ & $+$ & 15 & $+$ & $+$ & 23 & $+$ & $+$ \\
8 & $-$ & 0 & 16 & $-$ & $-$ & 24 & $-$ & $+$
\\ \hline
\end{tabular}
\end{center}
\begin{enumerate}[a.]
\item Why might a nonparametric statistical test be useful in comparing
drug A with placebo for this experiment.
\\
Why was it important to administer the placebo to the second eye of the same 
person rathe than to a different group of people with hay fever?
\item Compare the drug A and placebo eyes on redness and report an exact
p-value.  What procedure did you use?  Why?  Interpret. Also, perform a large
sample approximation and compare with the exact result. \\
Compare the drug A and placebo eyes on itching and report an exact p-value.
Also, perform a large sample approximation and compare with the exact result.
\end{enumerate}
\item Two entities (I and R) in rural India collect data on births report the
  data below for 1945.
  \begin{center}
    \begin{tabular}{ccc}
          & \multicolumn{2}{c}{I-list}  \\
R-list    & Present & Not \\ \hline
 Present  &  794    & 710 \\ 
    Not   &  741    &  -  \\ \hline
    \end{tabular}
  \end{center}
  \begin{enumerate}[a.]
  \item Assuming independence between the two groups, estimate the number
    of births in 1945 (give a confidence interval).
  \item Plot a likelihood function for the number of births in 1945.
  \end{enumerate}
\end{enumerate}
\end{document}
