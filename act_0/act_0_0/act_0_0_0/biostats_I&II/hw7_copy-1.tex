 \documentclass[12pt]{article}
\usepackage{geometry,amsmath,amssymb, graphicx, natbib, float, enumerate}
\geometry{margin=1in}
\renewcommand{\familydefault}{cmss}
\restylefloat{table}
\restylefloat{figure}

\newcommand{\code}[1]{\texttt{#1}}
\newcommand{\Var}{\mathrm{Var}}
\newcommand{\logit}{\mathrm{logit}}
\newcommand{\RQ}{[{\bf REQUIRED}]~}

\begin{document}
\noindent
{\bf BST 140.652 \\ Problem Set} \\

\begin{enumerate}[Problem 1.]
\item The table below gives the children's genders in a random sample
  of 1,000 two children families.
  \begin{center}
    \begin{tabular}{lcccccc}
             & \multicolumn{5}{c}{First child}  \\
             & \multicolumn{2}{c}{Male} && \multicolumn{2}{c}{Female} & \\ \cline{2-3} \cline{5-6}
Second child & Male & Female && Male & Female & Total \\ \hline
Count        &  218 & 227    && 278  & 277    & 1,000 \\ \hline
    \end{tabular}
  \end{center}
  \begin{enumerate}[a.]
  \item It is typically thought that the gender of offspring within a
    family are independent and identically distributed with males and
    females being equally likely.  Is this hypothesis supported by the
    data above?
  \item Specifically test independence of the gender of the first child to the second.
  \end{enumerate}
\item Consider the hypothesis testing problem of comparing two
  binomial probabilities $H_0: p_1 = p_2$.  Show that the square of
  statistic $(\hat{p}_1 - \hat{p}_2)/{\rm SE}_{\hat{p}_1 - \hat{p}_2}$
  is the same as the $\chi^2$ statistic. Here, the standard error in
  the denominator is calculated under the null hypothesis. (Clearly
  define any notation you introduce.)
\item A study of the effectiveness of {\sl streptokinase} in the
  treatment of patients who have been hospitalized after myocardial
  infarction involves a treated and control group.  In the
  streptokinase group, 2 of 15 patients died within 12 months.  In the
  control group, 4 of 19 died with 12 months.
  \begin{enumerate}[a.]
  \item Use Fisher's exact test to test for a difference in mortality
    rates.  Do this by hand by writing down all possible tables with
    fixed marginal totals.  You may confirm your results with a
    computer.
  \item Compare your results using the test statistics based on the
    normal and $\chi^2$ approximations.
\end{enumerate}
\item Download the class simulation data set ``task1.csv'' from the course
  web site. Here's the commands that I used to read it in
\begin{verbatim}
dat <- read.csv("task1.csv", header = FALSE)
dat2 <- dat[,1 : 10]
dat2 <- dat2[complete.cases(dat2),]
vec1 <- as.vector(unlist(dat2))
\end{verbatim}
  Dat is the orginal data. Dat2 contains only the data, removing any
  subjects containing errors. Vec1 is the data disregarding subject
  level information.
  \begin{enumerate}[a.]
  \item Do the numbers 1-10 appear to be equally likely? Perform
    the appropriate Chi-squared test.
  \item Approximate an exact Chi-squared test by doing the following.
    Simulate 1,000 random nultinomials under the null hypothesis with
    the command
\begin{verbatim}
simdat <- t(rmultinom(1000, size = length(vec1), p = rep(.1, 10)))
\end{verbatim}
    Obtain the chi-squared statistics for each with the command
\begin{verbatim}
chsqStats <- apply(simdat, 1, function(x) chisq.test(x)$statistic)
\end{verbatim}
    Calculate the percentage of time that these statistics are greater
    than the observed statistic. Explain how, provided the Monte Carlo
    sample is large, this is a P-value.
\end{enumerate}
\item A case-control study of esophageal cancer was performed.  Daily alcohol
consumption was ascertained $(80 +$ gm $=$ high, $0-79$ gm $=$ low). The data
was stratified by 3 age groups.
\begin{center}
\begin{tabular}{lllclllclll}
& \multicolumn{2}{c}{Alcohol} &&& \multicolumn{2}{c}{Alcohol}
&& &\multicolumn{2}{c}{Alcohol} \\
& H & L & && H & L &&& H & L \\
case & 8 & 5 && case & 25 & 21& & case & 50 & 61 \\
control & 52 & 164 && control & 29 & 138& & control & 27 & 208 \vspace{+0.25in} \\
& \multicolumn{2}{c}{Age 35-44} &&& \multicolumn{2}{c}{Age 45-54}
&&& \multicolumn{2}{c}{Age 55-64}
\end{tabular}
\end{center}
Assuming a constant odds ratio across age-strata, test to see if the
odds ratio is 1.  If not, estimate it.
\item Retinitis pigmentosa is a disease which manifests itself via
  different genetic modes of inheritance.  Cases have been documented
  with a dominant, recessive, and sex-linked form of inheritance.  It
  has been conjectured that the form of inheritance is related to the
  ethnic origin of the individual.  Cases of the disease have been
  surveyed in an English and Swiss population with the following
  results: out of 125 English cases, 46 had sex-linked disease, 25 had
  recessive disease and 54 had dominant disease; out of the 110 Swiss
  cases, one had sex-linked disease, 99 had recessive disease, and 10
  had dominant disease.  Based on these data is there a significant
  association between ethnic origin and genetic type?  Analyze and
  interpret (in words) this data. (10 points)
\item In a study of the association between cigarette smoking and lung
  cancer, 1,357 male lung cancer patients were compared with 1,357
  controls in terms of their cigarette consumption as follows:
\begin{center}
\begin{tabular}{lccccccc} \hline
& \multicolumn{6}{c}{Cigarette Consumption Daily} & \\
& $0$ & $1-$ & $5-$ & $15-$ & $25-$ & $50+$ & Total \\ \hline
Lung cancer patients & 7 & 49 & 516 & 445 & 299 & 41 & 1,357 \vspace{+0.10in}
\\
Controls & 61 & 91 & 615 & 408 & 162 & 20 & 1,357  \vspace{+0.05in} \\
\hline
\end{tabular}
\end{center}
Compute the odds ratio and log odds ratio in each of the 5 smoking groups
compared with non-smokers.  Find confidence intervals and graphically
display. Comment and interpret.  Can relative risks be estimated.  Why or
why not.
\item In a retrospective study of the possible effect of blood group on the
incidence of peptic ulcers, Woolf (1955) obtained data from three cities.
The table gives for each city data for blood groups 0 and A only.  In each
city, blood group is recorded for peptic ulcer subjects and for a control
series of individuals not having peptic ulcer.
\begin{center}
\begin{tabular}{lcccc} \hline
& \multicolumn{2}{c}{\underline{Peptic Ulcer}} & \multicolumn{2}{c}
{\underline{Control}} \\
& Group 0 & Group A & Group 0 & Group A \\ \hline
London & 911 & 579 & 4578 & 4219 \\
Manchester & 361 & 246 & 4532 & 3775 \\
Newcastle & 396 & 219 & 6598 & 5261 \vspace{+0.05in}  \\ \hline
\end{tabular}
\end{center}
\begin{enumerate}[a.]
\item  Compute the odds ratio for each city with a confidence interval.
Interpret.
\item Suppose that it is required to estimate P(ulcer$\mid$A) $-$ P(ulcer$\mid$0).
What further information is needed to do this from the current data?
\end{enumerate}
\item Suppose we wish to compare two treatments for breast cancer, viz., simple
mastectomy (S) and radical mastectomy (R).  We form matched pairs of women who
are within the same decade of age and with the same clinical condition to
receive the two treatments and measure their 5-year survival.  The results are
given (L$=$lived at least 5 years, D$=$died within 5 years) below.  Perform
an analysis of this data, and interpret your results.
\begin{center}
\tabcolsep=0.10in
\begin{tabular}{cccccc}
\\ \hline
& Treatment & Treatment && Treatment & Treatment \\
Pair & S Person & R Person & Pair & S Person & R Person \\ \hline
1 & L & L & 11 & D & D \\
2 & L & D & 12 & L & D \\
3 & L & L & 13 & L & L \\
4 & L & L & 14 & L & L \\
5 & L & L & 15 & L & D \\
6 & D & L & 16 & L & L \\
7 & L & L & 17 & L & D \\
8 & L & D & 18 & L & D \\
9 & L & D & 19 & L & L \\
10 & L & L & 20 & L & D \\ \hline
\end{tabular}
\end{center}
\item Suppose we are interested in comparing the effectiveness of 2 different
antibiotics A and B in treating gonorrhea.  We match each person receiving
antibiotic A with an equivalent person (age within 5 years, same sex) to whom we
give antibiotic B and we ask that these persons return to the clinic within 1 week
to see if the gonorrhea has been eliminated. \\
Suppose the results are as follows:
\begin{description}
\item For 40 pairs of people both antibiotics are successful.
\item For 20 pairs of people antibiotic A is effective while antibiotic B is not.
\item For 16 pairs of people antibiotic B is effective while antibiotic A is not.
\item For 3 pairs of people neither antibiotic is effective.
\end{description}
Perform an analysis to compare the relative effectiveness of the two
antibiotics.  Interpret your results.
\item Consider a retrospective study with matched pairs.  Show that McNemar's
test statistic is equivalent to performing a Mantel-Haenszel test for all
$2 \times 2$ tables (with one table for each pair).

\item A researcher is studying migration patterns. She collected
  the location of the current and previous homes for subjects who
  moved across regions. She recorded the following:
  \begin{center}
    \begin{tabular}{lrrr}
              & \multicolumn{3}{c}{Previous home} \\ \cline{2-4}
Current home & Northeast & Southeast & West \\ \hline
Northeast & -    & 267 & 255     \\
Southeast & 135  &  -  & 139     \\
West      & 240  & 234 &   -     \\ \hline
    \end{tabular}
  \end{center}
 Here the diagonals are not included since she only studied
 subjects who moved between regions. She would like
 to know if the probability of moving from region $a$ to $b$
 is the same as the probability of moving from region $b$ to $a$ for
 all regions $a$ and $b$.
 \begin{enumerate}[a.]
 \item Mathematically state her null and alternative hypotheses defining any notation you use.
 \item Calculate the expected counts under the null hypothesis.
 \item Perform the Chi-squared test and state your conclusions in the language of the problem. (Hint the df is $3$.)
 \end{enumerate}

\end{enumerate}

\end{document}
