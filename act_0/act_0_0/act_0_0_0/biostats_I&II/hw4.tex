\documentclass[12pt]{article}
\usepackage{geometry,amsmath,amssymb, graphicx, natbib, float, enumerate}
\geometry{margin=1in}
\renewcommand{\familydefault}{cmss}
\restylefloat{table}
\restylefloat{figure}

\newcommand{\code}[1]{\texttt{#1}}
\newcommand{\Var}{\mathrm{Var}}
\newcommand{\logit}{\mathrm{logit}}
\newcommand{\alone}{[{\bf ALONE}]~}

\begin{document}
\noindent
{\bf BST 140.651-652 \\ Problem Set 4} \\

\begin{enumerate}[Problem 1.]
\item A special study is conducted to test the hypothesis that
  persons with glaucoma have higher blood pressure than average. Two
  hundred subjects with glaucoma are recruited with a sample mean
  systolic blood pressure of $140mm$ and a sample standard deviation
  of $25mm$. (Do not use a computer for this problem.)
  \begin{enumerate}[a.]
  \item Construct a $95\%$ confidence interval for the mean systolic
    blood pressure among persons with glaucoma. Do you need to
    assume normality? Explain.
  \item If the average systolic blood pressure for persons without
    glaucoma of comparable age is $130mm$. Is there statistical
    evidence that the blood pressure is elevated?
  \end{enumerate}
\item  Consider the previous question.
  \begin{enumerate}[a.]
  \item Make a probabilistic argument that the interval
    $$\left[\bar X - z_{.95} \frac{s}{\sqrt{n}}, ~~\infty\right]$$
    is a 95\% {\em lower bound} for $\mu$.
  \end{enumerate}
\item  Suppose we wish to estimate the concentration $\mu$g/m$\ell$
  of a specific dose of ampicillin in the urine.  We recruit 25
  volunteers and find that they have sample mean concentration of
  7.0 $\mu$g/m\ $\ell$ with sample standard deviation 3.0 $\mu$
  g/m$\ell$.  Let us assume that the underlying population
  distribution of concentrations is normally distributed.
  \begin{enumerate}[a.]
  \item Find a 90\% confidence interval for the population mean concentration.
  \item How large a sample would be needed to insure that the length
    of the confidence interval is 0.5 $\mu$ g/m$\ell$ if it is assumed
    that the sample standard deviation remains at 3.0 $\mu$ g/m\ $\ell$?
  \end{enumerate}
\item Here we will verify that standardized means of iid normal data
  follow Gossett's $t$ distribution. Randomly generate $1,000 \times
  20$ normals with mean $5$ and variance $2$. Place these results in a
  matrix with $1,000$ rows. Using two \texttt{apply} statements on the
  matrix, create two vectors, one of the sample mean from each row and
  one of the sample standard deviation from each row. From these
  $1,000$ means and standard deviations, create $1,000$ $t$
  statistics.  Now use R's \texttt{rt} function to directly generate
  $1,000$ $t$ random variables with $19$ df. Use R's \texttt{qqplot}
  function to plot the quantiles of the constructed $t$ random
  variables versus R's $t$ random variables. Do the quantiles agree?
  Describe why they should.
\item Here we will verify the chi-squared result. Simulate $1,000$
  sample variances of $20$ observations from a normal distribution
  with mean $5$ and variance $2$.  Convert these sample variances so
  that they should be chi-squared random variables with $19$ degrees
  of freedom. Now simulate $1,000$ random chi-squared variables with
  $19$ degrees of freedom using R's \texttt{rchisq} function. Use R's
  \texttt{qqplot} function to plot the quantiles of the constructed
  chi-squared random variables versus those of R's random chi-squared
  variables. Do the quantiles agree? Describe why they should.
\item  If $X_1,\ldots,X_n$ are iid $N(\mu,\sigma^2)$ then we know that
  $(n - 1)S^2/\sigma^2$ is chi-squared with $n-1$ degrees of freedom.
  You were told that the expected value of a chi-squared is its
  degrees of freedom. Use this fact to verify the (already known fact)
  that $E[S^2] = \sigma^2$. (Note that $S^2$ is unbiased holds
  regardless of whether or not the data are normally distributed. Here
  we are just showing that the chi-squared result for normal data is
  not a contradiction of unbiasednes.)
\item A study of blood alcohol levels (mg/100 ml) at post mortem
  examination from traffic accident victims involved taking one blood
  sample from the leg, A, and another from the heart, B.  The results
  were:
\begin{center}
\ttfamily
\begin{tabular}{lrrrrr}
\hline
Case & A & B & Case & A & B   \\ \hline
1 &  44 &   44 &  11 & 265 &  277 \\
2 & 265 &  269 &  12 &  27 &   39 \\
3 & 250 &  256 &  13 &  68 &   84 \\
4 & 153 &  154 &  14 & 230 &  228 \\
5 &  88 &  83  &  15 & 180 &  187 \\
6 & 180 &  185 &  16 & 149 &  155 \\
7 &  35 &   36 &  17 & 286 &  290 \\
8 & 494 &  502 &  18 &  72 &  80  \\
9 & 249 &  249 &  19 &  39 &  50  \\
10& 204 &  208 &  20 & 272 &  271 \\ \hline
\end{tabular}
\end{center}
\normalfont
\begin{enumerate}[a.]
\item Create a graphical display comparing a case's blood alcohol level in the heart to that in the leg.  Comment on any interesting patterns from your graph.
\item Create a graphical display of the distribution of the difference in blood alcohol levels between the heart and the leg.
\item Do these results indicate that in general blood alcohol levels may differ between samples taken from the leg and the heart? Give confidence interval and
  interpret your results.
\item Create a profile likelihood for the true mean difference and
  interpret.
\item Create a likelihood for the variance of the difference in
  alcohol levels and interpret.
\end{enumerate}
\item A random sample was taken of 20 patients admitted to a hospital
  with a certain diagnosis.  The lengths of stays in days for the 20
  patients were
\begin{center}
\begin{tabular}{cccccccccc}
 4, & 2, & 4, & 7, & 1, & 5, & 3, & 2, & 2, & 4 \vspace{+0.15in} \\
5, & 2, & 5, & 3, & 1, & 4, & 3, & 1, & 1, & 3
\end{tabular}
\end{center}
\begin{enumerate}[a.]
\item Calculate a 95\% confidence interval (use the method
$\overline{X} \pm t \;\;$ SE) for the mean length of hospital stay.  Is your answer reasonable? What underlying assumptions were required for this method and are they reasonable?
\item Calculate a 95\% percentile bootstrap interval and interpret.
\end{enumerate}
\item Refer to the previous problem. Take logs of the data (base ``e'')
  \begin{enumerate}[a.]
  \item  Calculate a 95\% confidence interval for the mean of the log length of stay.
  \item Take antilogs (exponential) of the endpoints of the
    confidence interval found in part a..  Explain why that is a
    95\% confidence interval for the median length of stay if the data
    is lognormally distributed (lognormally distributed is when the
    logarithm of the data points has a normal distribution).
    Technically, under the lognormal assumption, is the confidence
    interval that you found in this equation also a confidence interval for
    the mean length of stay?
  \end{enumerate}
\item  Let $p$ denote the unknown proportion of rocks in a riverbed
  that are sedimentary in type.  Suppose that $X = 12$ of a sample of
  $n = 20$ rocks collected in random locations are found to be
  sedimentary in type.
\begin{enumerate}[a.]
\item Plot the likelihood for the parameter $p$ and interpret.
\item From your graphs, determine the value of $\hat{p}$ of $p$ where
  the curve reaches its maximum.  Does this value for the maximum make
 intuitive sense?  Comment in one or two sentences.
\item Show that the point that maximizes the binomial likelihood is always $X/n$.
\item Use the CLT to create a confidence interval for the true
  proportion of rocks that are sedimentary. Interpret your results.
\item A much larger study is planned and the researchers would like to
  know how large $n$ should be to have a margin of error on the
  estimate for the proportion of sedimentary rocks that is no larger
  than $.01$ for a $95\%$ confidence interval? Use the fact that $p(1
  - p) \leq 1/4$. Also try the calculation with the estimate of $p$
  from the current study.
\end{enumerate}
\item This problem investigates the performance of the Wald confidence interval
\begin{enumerate}[a.]
\item Using a computer, generate $1000$ Binomial random variables for $n=10$ and $p =.3$ Calculate the percentage of times that
$$
\hat p \pm 1.96\sqrt{\hat p (1 - \hat p) / n}
$$
contains the true value of p. Here $\hat p = X/n$ where $X$ is each
binomial variable. Do the intervals appear to have the coverage that
they are supposed to?
\item Repeat the calculation only now use the interval
$$
\tilde p \pm 1.96\sqrt{\tilde p (1 - \tilde p) / \tilde{n}}
$$
where $\tilde{n} = n+4$ and $\tilde p = (X + 2) / (n + 4)$. Does the coverage appear to be
closer to .95?
\item Repeat this comparison (parts a. - d.) for $p = .1$ and $p =
  .5$. Which of the two intervals appears to perform better?
\end{enumerate}
\item Forced expiratory volume FEV is a standard measure of pulmonary
  function.  We would expect that any reasonable measure of pulmonary
  function would reflect the fact that a person's pulmonary function
  declines with age after age 20.  Suppose we test this hypothesis by
  looking at 10 nonsmoking males ages 35-39, heights 68-72 inches and
  measure their FEV initially and then once again 2 years later.  We
  obtain this data.
\begin{center}
\begin{tabular}{cccccc}
\hline
& Year 0 & Year 2 && Year 0 & Year 2 \\
& FEV & FEV && FEV & FEV \\
Person & (L) & (L) & Person & (L) & (L) \vspace{+0.05in} \\ \hline
1 & 3.22 & 2.95 & 6 & 3.25 & 3.20 \\
2 & 4.06 & 3.75 & 7 & 4.20 & 3.90 \\
3 & 3.85 & 4.00 & 8 & 3.05 & 2.76 \\
4 & 3.50 & 3.42 & 9 & 2.86 & 2.75 \\
5 & 2.80 & 2.77 & 10 & 3.50 & 3.32 \vspace{+0.05in} \\ \hline
\end{tabular}
\end{center}
\begin{enumerate}[a.]
\item Create the relevant confidence interval and interpret.
\item Create the relevant profile likelihood and interpret.
\item Create a likelihood function for the variance of the change in
  FEV.
\end{enumerate}
\item Another aspect of the preceding study involves looking at the
  effect of smoking on baseline pulmonary function and on change in
  pulmonary function over time.  We must be careful since FEV depends
  on many factors, particularly age and height.  Suppose we have a
  comparable group of 15 men in the same age and height group who are
  smokers and we measure their FEV at year 0.  The data are given (For
  purposes of this exercise assume equal variance where appropriate).
\begin{center}
\begin{tabular}{cccccc} \hline
& FEV & FEV && FEV & FEV \\
& Year 0 & Year 2 && Year 0 & Year 2 \\
Person & (L) & (L) & Person & (L) & (L) \vspace{+0.05in} \\ \hline
1 & 2.85 & 2.88 & 9 & 2.76 & 3.02 \\
2 & 3.32 & 3.40 & 10 & 3.00 & 3.08 \\
3 & 3.01 & 3.02 & 11 & 3.26 & 3.00 \\
4 & 2.95 & 2.84 & 12 & 2.84 & 3.40 \\
5 & 2.78 & 2.75 & 13 & 2.50 & 2.59 \\
6 & 2.86 & 3.20 & 14 & 3.59 & 3.29 \\
7 & 2.78 & 2.96 & 15 & 3.30 & \ 3.32 \\
8 & 2.90 & 2.74 &&& \vspace{+0.05in} \\ \hline
\end{tabular}
\end{center}
\begin{enumerate}[a.]
\item Graphically display the distribution of change in pulmonary
  function for smokers and nonsmokers (from the previous problem).
\item Calculate a confidence interval to determine if there is
  evidence to suggest that the change in pulmonary function over 2
  years is the same in the two groups.  State your assumptions and
  interpret your results.
\end{enumerate}
\item  In a trial to compare a stannous fluoride dentifrice A, with a
  commericially available fluoride free dentifrice D, 260 children
  received A and 289 received D for a 3-year period.  The mean DMFS
  increments (the number of new Decayed Missing and Filled tooth
  Surfaces) were 9.78 with standard deviation 7.51 for A and 12.83
  with standard deviation 8.31 for D.  Is this good evidence that, in
  general, one of these dentifrices is better than the other at
  reducing tooth decay?  If so, within what limits would the average
  annual difference in DMFS increment be expected to be?
\item Suppose that $18$ obese subjects were randomized, $9$ each, to a
  new diet pill and a placebo. Subjects' body mass indices (BMIs) were
  measured at a baseline and again after having received the treatment
  or placebo for four weeks.  The average difference from follow-up to
  the baseline (followup - baseline) was $-3$ $kg/m^2$ for the treated
  group and $1$ $kg/m^2$ for the placebo group.  The corresponding
  standard deviations of the differences was $1.5$ $kg / m^2$ for the
  treatment group and $1.8$ $kg / m^2$ for the placebo group. Does the
  change in BMI over the two year period appear to differ between the
  treated and placebo groups? (Show some work and interpret your
  results.) Assume normality and a
  common variance.
\item In a random sample of $100$ subjects with low back pain, $27$
  reported an improvement in symptoms after execise therapy. Give and
  interpret an interval estimate for the
  true proportion of subjects who respond to exercise therapy.
  \item Suppose that systolic blood pressures were taken on $16$
oral contraceptive users and $16$ controls at baseline and again then
two years later. The average difference from follow-up SBP to the
baseline (followup - baseline) was $11$ $mmHg$ for oral contraceptive
users and $4$ $mmHg$ for controls.  The corresponding standard
deviations of the differences was $20$ $mmHg$ for OC users and $28$
$mmHg$ for controls.
\begin{enumerate}[a.]
\item Calculate and interpret a $95\%$ confidence interval for the
  change in systolic blood pressure for oral contraceptive users;
  assume
  normality.
\item Does the change in SBP over the two year period appear to differ
  between oral contraceptive users and controls? Create the relevant
  $95\%$ confidence interval and interpret.  Assume normality and a
  common variance.
\end{enumerate}

\end{enumerate}
\end{document}
