\documentclass[12pt]{article}
\usepackage{geometry,amsmath,amssymb, graphicx, natbib, float, enumerate}
\geometry{margin=1in}
\renewcommand{\familydefault}{cmss}
\restylefloat{table}
\restylefloat{figure}

\newcommand{\code}[1]{\texttt{#1}}
\newcommand{\Var}{\mathrm{Var}}
\newcommand{\logit}{\mathrm{logit}}
\newcommand{\RQ}{[{\bf REQUIRED}]~}

\begin{document}
\noindent
{\bf BST 140.652 \\ Problem Set 5} \\

\begin{enumerate}[Problem 1.]
\item A laboratory experiment found that in a random sample of $20$
  frog eggs having aquaporins, $17$ exploded when put into water.
  \begin{enumerate}[a.]
  \item Plot and interpret the posteriors for $p$ assuming a beta prior with
    parameters $(2,2)$, $(1, 1)$ and $(.5, .5)$.
  \item Calculate and interpret the credible interval for each of the
    beta prior parameter esttings. Note that the R package
    \texttt{binom} may be of use.
  \end{enumerate}
\item A study of blood alcohol levels (mg/100 ml) at post mortem
  examination from traffic accident victims involved taking one blood
  sample from the leg, A, and another from the heart, B.  The results
  were:
\begin{center}
\ttfamily
\begin{tabular}{lrrrrr}
\hline
Case & A & B & Case & A & B   \\ \hline
1 &  44 &   44 &  11 & 265 &  277 \\
2 & 265 &  269 &  12 &  27 &   39 \\
3 & 250 &  256 &  13 &  68 &   84 \\
4 & 153 &  154 &  14 & 230 &  228 \\
5 &  88 &  83  &  15 & 180 &  187 \\
6 & 180 &  185 &  16 & 149 &  155 \\
7 &  35 &   36 &  17 & 286 &  290 \\
8 & 494 &  502 &  18 &  72 &  80  \\
9 & 249 &  249 &  19 &  39 &  50  \\
10& 204 &  208 &  20 & 272 &  271 \\ \hline
\end{tabular}
\end{center}
\normalfont Test whether or not the mean blood alcohol level differs
between the heart and the leg.  Give the appropriate null and
alternative hypotheses. Give the relevant P-value. Interpret your
results, state your assumptions.
\item Forced expiratory volume FEV is a standard measure of pulmonary
  function.  We would expect that any reasonable measure of pulmonary
  function would reflect the fact that a person's pulmonary function
  declines with age after age 20.  Suppose we test this hypothesis by
  looking at 10 nonsmoking males ages 35-39, heights 68-72 inches and
  measure their FEV initially and then once again 2 years later.  We
  obtain this data.
\begin{center}
\begin{tabular}{cccccc}
\hline
& Year 0 & Year 2 && Year 0 & Year 2 \\
& FEV & FEV && FEV & FEV \\
Person & (L) & (L) & Person & (L) & (L) \vspace{+0.05in} \\ \hline
1 & 3.22 & 2.95 & 6 & 3.25 & 3.20 \\
2 & 4.06 & 3.75 & 7 & 4.20 & 3.90 \\
3 & 3.85 & 4.00 & 8 & 3.05 & 2.76 \\
4 & 3.50 & 3.42 & 9 & 2.86 & 2.75 \\
5 & 2.80 & 2.77 & 10 & 3.50 & 3.32 \vspace{+0.05in} \\ \hline
\end{tabular}
\end{center}
\begin{enumerate}[a.]
\item Preform and interpret the relevant test.  Give
  the appropriate null and alternative hypotheses. Intepret your
  results, state your assumptions and give a P-value.
\item A large test comparing the two-year decline in non-smokers of a
  different age.  Perform a sample size calculation to detect a change
  in FEV over two years at least as large as that detected for males
  age 35-39. Use the data above a for any
  relevant constants that you might need.
\end{enumerate}
\item Another aspect of the preceding study involves looking at the
  effect of smoking on baseline pulmonary function and on change in
  pulmonary function over time.  We must be careful since FEV depends
  on many factors, particularly age and height.  Suppose we have a
  comparable group of 15 men in the same age and height group who are
  smokers and we measure their FEV at year 0.  The data are given (For
  purposes of this exercise assume equal variance where appropriate).
\begin{center}
\begin{tabular}{cccccc} \hline
& FEV & FEV && FEV & FEV \\
& Year 0 & Year 2 && Year 0 & Year 2 \\
Person & (L) & (L) & Person & (L) & (L) \vspace{+0.05in} \\ \hline
1 & 2.85 & 2.88 & 9 & 2.76 & 3.02 \\
2 & 3.32 & 3.40 & 10 & 3.00 & 3.08 \\
3 & 3.01 & 3.02 & 11 & 3.26 & 3.00 \\
4 & 2.95 & 2.84 & 12 & 2.84 & 3.40 \\
5 & 2.78 & 2.75 & 13 & 2.50 & 2.59 \\
6 & 2.86 & 3.20 & 14 & 3.59 & 3.29 \\
7 & 2.78 & 2.96 & 15 & 3.30 & \ 3.32 \\
8 & 2.90 & 2.74 &&& \vspace{+0.05in} \\ \hline
\end{tabular}
\end{center}
Test the hypothesis that the change in FEV is equivalent between
non-smokers and smokers. State relevant assumptions and interpret
your result. Give the relevant P-value.
\item Perform the following simulation. Randomly simulate $1,000$
  sample means of size $16$ from a normal distribution with means $5$
  and variances $1$. Calculate $1,000$ test statistics for a test of
  $H_0:\mu = 5$ versus $H_a:\mu< 5$. Using these test statistics calculate
  $1,000$ P-values for this test. Plot a histogram of the P-values. Note, this
  exercise demonstrates the interesting fact that the distribution of P-values is
  uniform.
\item  Suppose that systolic blood pressures were taken on $16$
oral contraceptive users and $16$ controls at baseline and again then
two years later. The average difference from follow-up SBP to the
baseline (followup - baseline) was $11$ $mmHg$ for oral contraceptive
users and $4$ $mmHg$ for controls.  The corresponding standard
deviations of the differences was $20$ $mmHg$ for OC users and $28$
$mmHg$ for controls.
\begin{enumerate}[a.]
\item Calculate and interpret a $95\%$ confidence interval for the
  {\bf relative} change in systolic blood pressure for oral
  contraceptive users; assume normality.
\item Does the change in SBP over the two year period appear to differ
  between oral contraceptive users and controls? Perform the relevant
  hypothesis test and interpret. Give a P-value. Assume normality and
  a common variance.
\end{enumerate}
\item Will a Student's $T$ or $Z$ hypothesis test for a mean with
  the data recorded in pounds always agree with the same test
  conducted on the same data recorded in kilograms? (explain)
\item A researcher consulting you is very concerned about falsely
  rejecting her null hypothesis. As a result the researcher decides to
  increase the sample size of her study. Would you have anything to
  say?  (explain)
\item Researchers studying brain volume found that in a random sample
  of 16 sixty five year old subjects with Alzheimer's disease, the average
  loss in grey matter volume as a person aged four years was .1 $mm^3$
  with a standard deviation of .04 $mm^3$.
\begin{enumerate}[a.]
\item Calculate and interpret a P-value for the
  hypothesis that there is no loss in grey matter volumes as
  people age. Show your work.
\item The researchers would now like to plan a similar
    study in 100 healthy adults to detect a four year mean loss of .01
    $mm^3$. Motivate a general formula for power calculations in this
    setting and calculate the power for a test with $\alpha = .05$?
    Assume that the variation in grey matter loss will be similar to
    that estimated in the Alzheimer's study.
\end{enumerate}
\item A recent Daily Planet article reported on a study of a two week
  weight loss program.  The study reported a 95\% confidence interval
  for weight loss from baseline of [2 lbs, 6 lbs]. (There was no
  control group, all subjects were on the weight loss program.)  The
  exact sample size was not given, though it was known to be over 200.
  \begin{enumerate}[a.]
  \item  What can be said of a $\alpha = 5\%$ hypothesis
    test of whether or not there was any weight change from baseline?
    Can you determine the result of a $\alpha = 10\%$ test without any
    additional calculation or information? (explain your answer)
  \end{enumerate}
\item Suppose that $18$ obese subjects were randomized, $9$ each, to a
  new diet pill and a placebo. Subjects' body mass indices (BMIs) were
  measured at a baseline and again after having received the treatment
  or placebo for four weeks.  The average difference from follow-up to
  the baseline (followup - baseline) was $-3$ $kg/m^2$ for the treated
  group and $1$ $kg/m^2$ for the placebo group.  The corresponding
  standard deviations of the differences was $1.5$ $kg / m^2$ for the
  treatment group and $1.8$ $kg / m^2$ for the placebo group.
  Does the change in BMI over the two year period appear to differ
  between the treated and placebo groups? Perform the relevant
  test and interpret. Give a P-value.  Assume normality and a
  common variance.
\end{enumerate}
\end{document}
