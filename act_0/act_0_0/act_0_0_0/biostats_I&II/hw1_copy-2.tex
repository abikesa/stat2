\documentclass[12pt]{article}
\usepackage{geometry,amsmath,amssymb, graphicx, natbib, float, enumerate}
\geometry{margin=1in}
\renewcommand{\familydefault}{cmss}
\restylefloat{table}
\restylefloat{figure}

\newcommand{\code}[1]{\texttt{#1}}
\newcommand{\Var}{\mathrm{Var}}
\newcommand{\logit}{\mathrm{logit}}

\begin{document}
\noindent
{\bf BST 140.651 \\ Problem Set 1} \\

\begin{enumerate}[Problem 1.]
\item Show the following
\begin{enumerate}[a.]
\item  $P(\emptyset) = 0$.
\item  $P(E) = 1 - P(E^c)$.
\item  If $A \subset B$ then $P(A) \leq P(B)$.
\item  For any $A$ and $B$, $P(A \cup B) = P(A) + P(B) - P(A \cap B)$.
\item  $P(A \cup B) = 1 - P(A^c \cap B^c)$.
\item   $P(A \cap B^c) = P(A) - P(A \cap B)$.
\item  $P(\cup_{i=1}^n E_i) \leq \sum_{i=1}^n P(E_i)$.
\item  $P(\cup_{i=1}^n E_i) \geq \max_i P(E_i)$.
\end{enumerate}
\item Cryptosporidium is a pathogen that can cause gastrointestinal illness with diarrhea; infections can lead to death in individuals with a weakened
immune system. During a recent outbreak of cryptosporidiosis in 21\%
of
  two parent families at least one of the parents has contracted the
  disease.  In 9\% of the families the father has contracted
  cryptosporidiosis while in 5\% of the families both the mother and father
  have contracted cryptosporidiosis.
  \begin{enumerate}[a.]
  \item What event does the probability one minus the probability that
    both have contracted cryptosporidiosis represent?
  \item What's the probability that either the mother or the father has contracted
  cryptosporidiosis?
  \item What's the probability that the mother has contracted cryptosporidiosis but the father has not?
  \item What's the probability that the mother has contracted cryptosporidiosis?
  \item What's the probability that neither the mother nor the father has contracted cryptosporidiosis?
  \item What's the probability that the mother has contracted cryptosporidiosis but the father has not?
  \end{enumerate}
\item Suppose $h(x)$ is such that $h(x) > 0$ for $x=1,2,\ldots,I$.
  Argue that $ p(x) = h(x) / \sum_{i=1}^I h(i) $ is a valid pmf.
\item  Suppose a function $h$ is such that $h>0$ and $c = \int_{-\infty}^\infty h(x) dx < \infty$.
  Show that $f(x) = h(x) / c$ is a valid density.
\item Suppose that, for a randomly drawn subject from a particular
  population, the proportion of a their skin that is covered in
  freckles follows a density that is constant on $[0,1]$. (This is
  called the {\bf uniform density on $[0,1]$}.) That is, $f(x) = k$ for $0\leq x
  \leq 1$.
  \begin{enumerate}[a.]
  \item Draw this density. What must $k$ be?
  \item Suppose a random variable, $X$, follows a uniform
    distribution. What is the probability that $X$ is between .1 and
    .7? Interpret
    this probability in the context of the problem.
  \item Verify the previous calculation in R. What's the probability that $a < X < b$ for
    generic values $0 < a < b < 1$?
  \item What is the distribution function associated with this density?
  \item What is the median of this density? Interpret the median in the context of the problem.
  \item What is the $95^{th}$ percentile? Interpret this percentile in the context of the problem.
  \item Do you believe that the proportion of freckles on subjects in a
    given population could feasibly follow this distribution? (Why or
    why not.)
  \end{enumerate}
\item Let $U$ be a continuous random variable with a uniform density on
$[0,1]$ and $F(\cdot)$ be any strictly increasing cdf.
  \begin{enumerate}[a.]
  \item Show that $F^{-1}(U)$ is a random variable with cdf equal to
  F.
  \item Describe a simulation procedure in R that can simulate any
  iid sample from a distribution with a given cdf $F(\cdot)$
  \item Simulate $100$ Normal iid variables using only simulations from
  a uniform distribution and the normal cdf in R
  \end{enumerate}
\item Let $0 \leq \pi \leq 1$ and $f_1$ and $f_2$ be two continuous
  densities with associated distribution functions $F_1$ and $F_2$ and
  survival functions $S_1$ and $S_2$. Let $g(x) = \pi f_1(x) + (1 - \pi)f_2(x)$.
    \begin{enumerate}[a.]
    \item Show that $g$ is a valid density.
    \item Write the distribution function associated with $g$ in the terms of
    $F_1$ and $F_2$.
    \item Write the survival function associated with $g$ in the terms of $S_1$ and $S_2$.
    \end{enumerate}
\item  Radiologists have created cancer risk summary that, for a given
  population of subjects, follows (a specific instance of) the {\bf logistic} density
  $$
  \frac{e^{-x}}{(1 + e^{-x})^2} ~~~~~~~\mbox{for}~ -\infty < x < \infty.
  $$
  \begin{enumerate}[a.]
  \item Show that this is a valid density.
  \item Calculate the distribution function associated with this density.
  \item What value do you get when you plug $0$ into the distribution function? Interpret this
    result in the context of the problem.
  \item Define the {\em odds} an event with probability $p$
    as $p / (1 - p)$. Prove that the $p^{th}$ quantile from this
    distribution is $\log\{p / (1 - p)\}$; which is the natural log of
    the odds of an event with probability $p$.
  \end{enumerate}
\item Quality control experts estimate that the time (in years) until a specific electronic
  part from an assembly line fails follows (a specific instance of) the {\bf Pareto}
  cdf
$$
F(x)=\left\{
\begin{array}{lll}
1-\left(\frac{x_0}{x}\right )^\alpha &\mbox{for}& x\geq x_0\\
0&\mbox{for}& x< x_0 \end{array}\right.
$$
The parameter $x_0$ is called the scale parameter, while $\alpha$ is
the shape or tail index parameter. The distribution is often denoted
by ${\rm Pa}(x_0,\alpha)$.
\begin{enumerate}[a.]
\item Derive the density of the Pareto distribution.
\item Plot the density and the cdf for $x_0=1,2,5$ and
$\alpha=0.1,1,10$. Comment on the interpretation.
\item Generate Pareto random variables using simulated uniform random variables in R.
\item What is the survival function associated with this density?
  Interpret a value (say $x=10$ years for $\alpha=1$ and $x_0=2$ ) evaluated in the survival function in the context of the problem.
\item Find the $p^{th}$ quantile for this density. For $p=.8$ interpret this
  value in the context of the problem.
\end{enumerate}
\item  Suppose that a density is of the form $cx^k$ for some constant $k > 1$ and $0 < x < 1$.
  \begin{enumerate}[a.]
  \item Find $c$.
  \item Find the cdf.
  \item Derive a formula for the $p^{th}$ quantile from $f$.
  \item Let $0 \leq a < b \leq 1$. Derive a formula for $P(a < X < b)$.
  \end{enumerate}
\item Suppose that the time in days until hospital discharge for a certain patient population
follows a density $f(x) = c\exp(-x/2.5)$ for $x > 0$.
\begin{enumerate}[a.]
    \item What value of $c$ makes this a valid density?
    \item Find the distribution function for this density.
    \item Find the survival function.
    \item Calculate the probability that a person takes longer than 11 days to be discharged.\
    \item What is the median number of days until discharge?
\end{enumerate}
\item The (lower) incomplete gamma function is
    defined as $\Gamma(k, c) = \int_{0}^c x^{k-1}\exp(-x)dx$.
    By convention $\Gamma(k, \infty)$, the complete gamma function, is written $\Gamma(k)$.
    Consider a density
    $$
    \frac{1}{\Gamma(\alpha)} x^{\alpha - 1} \exp(-x) ~~ \mbox{for} ~~ x > 0
    $$
    where $\alpha$ is a known number.
    \begin{enumerate}[a.]
    \item Argue that this is a valid density.
    \item Write out the survival function associated with this density using gamma functions
    \item Let $\beta$ be a known number; argue that
    $$
    \frac{1}{\beta^\alpha\Gamma(\alpha)} x^{\alpha - 1} \exp(-x/\beta) ~~ \mbox{for} ~~ x > 0
    $$
    is a valid density. This is known as the {\bf gamma density}.
    \item Plot the Gamma density for different values of $\alpha$ and $\beta$.
    \end{enumerate}
\item The {\bf Weibull density} is useful in survival analysis. Its form is given by
$$
\frac{\gamma}{\beta}x^{\gamma - 1}\exp\left(-x^\gamma / \beta\right),
$$
for $x > 0$ and $\gamma$ and $\beta$ are fixed known numbers.
\begin{enumerate}[a.]
    \item Demonstrate that the Weibull density is a valid density.
    \item Calculate the survival function associated with the Weibull density.
    \item Calculate the median of the Weibull density.
    \item Plot the Weibull density for different values of $\gamma$ and $\beta$.
\end{enumerate}
\item The Beta function is given by $B(\alpha, \beta) =  \int_0^1 x^{\alpha-1} (1 - x)^{\beta - 1}$
for $\alpha > 0$ and $\beta > 0$ . It turns out that
$$
B(\alpha, \beta) = \Gamma(\alpha)\Gamma(\beta)/\Gamma(\alpha + \beta).
$$
The {\bf Beta density} is given by $\frac{1}{B(\alpha,\beta)}x^{\alpha-1}(1 - x)^{\beta-1}$ for fixed
$\alpha > 0$ and $\beta > 0$. This density is useful for
\begin{enumerate}[a.]
\item Argue that the Beta density is a valid density.
\item Argue that the uniform density is a special case of the beta density.
\item Plot the beta density for different values of $\alpha$ and $\beta$.
\end{enumerate}
\item A famous formula is $e^{\lambda} = \sum_{x=0}^\infty \frac{\lambda^{x}}{x!}$ for any value
    of $\lambda$. Assume that the count of the number of people infected
with a particular disease per year follows a mass function given by
$$
P(X = x) = \frac{e^{-\lambda} \lambda^x}{x!} ~~ \mbox{for} ~~ x = 0, 1, 2, 3, \ldots
$$
where $\lambda$ is a fixed known number. (This is know as the {\bf Poisson mass function}.)
\begin{enumerate}[a.]
\item Argue that $\sum_{x=0}^\infty P(X = x) = 1$.
\end{enumerate}
\item Consider counting the number of coin flips from an unfair coin with success probability $p$ until a head is obtained, say $X$. The mass
function for this process is given by $P(X = x) = p(1 - p)^{x-1}$ for $x = 1, 2, 3, \ldots$.
This is called the {\bf geometric mass function}.
\begin{enumerate}[a.]
\item Argue mathematically that this is a valid probability mass function. Hint, the geometric series is given by $\frac{1}{1-r} = \sum_{k=0}^\infty r^k$ for $|r| < 1$.
\item Calculate the survival distribution $P(X > x)$ for the geometric distribution for
integer values of $x$.
\end{enumerate}
\end{enumerate}
\end{document}
