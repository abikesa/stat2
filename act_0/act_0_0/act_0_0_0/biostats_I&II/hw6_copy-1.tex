\documentclass[12pt]{article}
\usepackage{geometry,amsmath,amssymb, graphicx, natbib, float, enumerate}
\geometry{margin=1in}
\renewcommand{\familydefault}{cmss}
\restylefloat{table}
\restylefloat{figure}

\newcommand{\code}[1]{\texttt{#1}}
\newcommand{\Var}{\mathrm{Var}}
\newcommand{\logit}{\mathrm{logit}}
\newcommand{\RQ}{[{\bf REQUIRED}]~}

\begin{document}
\noindent
{\bf BST 140.652 \\ Problem Set 6} \\

\begin{enumerate}[Problem 1.]
\item Consider the hypothesis testing problem of comparing two
  binomial probabilities $H_0: p_1 = p_2$.  Show that the square of
  statistic $(\hat{p}_1 - \hat{p}_2)/{\rm SE}_{\hat{p}_1 - \hat{p}_2}$
  is the same as the $\chi^2$ statistic. Here, the standard error in
  the denominator is calculated under the null hypothesis. (Clearly
  define any notation you introduce.)
\item A study of the effectiveness of {\sl streptokinase} in the
  treatment of patients who have been hospitalized after myocardial
  infarction involves a treated and control group.  In the
  streptokinase group, 20 of 150 patients died within 12 months.  In the
  control group, 40 of 190 died with 12 months.
  \begin{enumerate}[a.]
  \item Test equivalence of the two proportions.
  \item Give confidence intervals for the absolute change in proportions, the relative risk and odds ratio.  
  \item Create Bayesian credible intervals for the risk difference,
    risk ratio and odds ratio. Plot the posterior for each and
    interpret the results.
  \end{enumerate}
\item Researchers are interested in estimating the natural log of the
  proportion of people in the population with hypertension. In a
  random sample of $n$ subjects, let $X$ be the number with
  hypertension. Derive a confidence interval for the natural log of the
  proportion of people with hypertension. Assume that $n$ is
  large. 
\item This problem considers the delta method.
  \begin{enumerate}[a.]
  \item Derive the asymptotic standard error for $\sqrt{\hat p}$
    where $\hat p$ is a binomial sample proportion.
  \item Assume that $n = 200$ and $p = .5$. Implement a simulation
    study to verify that the delta method results in approximately
    normally distributed variables.
  \end{enumerate}
\item In this homework, we will evaluate the performance of the log
  odds ratio interval estimate
  $$
  \log \hat{OR} \pm 1.96 \sqrt{1/n_{11} + 1/n_{12} + 1 / n_{21} + 1 / n_{22}}.
  $$
  Use R to generate $1,000$ random binomials with \texttt{n1} trials
  and \texttt{p1} success probability; call this vector \texttt{x}.
  Use R to generate $1,000$ random binomials with \texttt{n2} trials
  and \texttt{p2} success probability; call this vector \texttt{y}.
  Squash these to vectors together with the command \texttt{z =
    cbind(x, y)}. Now, create $1,000$ sample odds ratios with the
  command
\begin{verbatim}
OR = apply(z, 1, 
           function(x) x[1] * (n2 - x[2]) / (x[2] * (n1 - x[1]))
          )
\end{verbatim}
  Log these odds ratios to obtain $1,000$ sample log odds ratios. Now
  obtain $1,000$ standard errors with the command
\begin{verbatim}
SELOGOR = apply(z, 1, 
           function(x) sqrt(1 / x[1] + 1 / (n1 - x[1]) + 
                            1 / x[2] + 1 / (n2 - x[2])
                           )
          )
\end{verbatim}
  Now, see how often the interval for the log odds ratio contains the
  true log odds ratio.  Repeat this process for all of the following
  combinations
\begin{verbatim}
p1 = .1; p2 = .1; n1 = 100; n2 = 100
p1 = .1; p2 = .5; n1 = 100; n2 = 100
p1 = .1; p2 = .9; n1 = 100; n2 = 100
p1 = .5; p2 = .5; n1 = 100; n2 = 100
p1 = .5; p2 = .9; n1 = 100; n2 = 100
p1 = .9; p2 = .9; n1 = 100; n2 = 100
\end{verbatim}
  Summarize your findings. 
\item In this homework, we will also evaluate the performance of the log
  relative risk interval estimate
  $$
  \log \hat{RR} \pm 1.96 \sqrt{\hat (1 - \hat p_1) / (\hat p_1 n_1) + (1 - \hat p_2) / (\hat p_2 n_2)} 
  $$
  Use R to generate $1,000$ random binomials with \texttt{n1} trials
  and \texttt{p1} success probability; call this vector \texttt{x}.
  Use R to generate $1,000$ random binomials with \texttt{n2} trials
  and \texttt{p2} success probability; call this vector \texttt{y}.
  Squash these to vectors together with the command \texttt{z =
    cbind(x, y)}.  Now, create $1,000$ sample risk ratios with the
  command
\begin{verbatim}
RR = apply(z, 1, 
           function(x) (x[1] / n1) / (x[2] / n2)
          )
\end{verbatim}
  Log these risk ratios to obtain $1,000$ sample log risk ratios. Now
  obtain $1,000$ standard errors with the command
\begin{verbatim}
SELOGRR = apply(z, 1, 
             function(x) {
                phat1 <- x[1] / n1
                phat2 <- x[2] / n2
                sqrt((1 - phat1) / phat1 / n1 + (1 - phat2) / phat2 / n2)
             }
          )
\end{verbatim}
  Now, see how often the interval for the log relative risk contains the
  true log relative risk. Repeat this process for the following combinations
\begin{verbatim}
p1 = .1; p2 = .1; n1 = 100; n2 = 100
p1 = .1; p2 = .5; n1 = 100; n2 = 100
p1 = .1; p2 = .9; n1 = 100; n2 = 100
p1 = .5; p2 = .5; n1 = 100; n2 = 100
p1 = .5; p2 = .9; n1 = 100; n2 = 100
p1 = .9; p2 = .9; n1 = 100; n2 = 100
\end{verbatim}
Summarize your findings.
% \item Download the class simulation data set ``task1.csv'' from the course
%   web site. Here's the commands that I used to read it in 
% \begin{verbatim}
% dat <- read.csv("task1.csv", header = FALSE)
% dat2 <- dat[,1 : 10]
% dat2 <- dat2[complete.cases(dat2),]
% vec1 <- as.vector(unlist(dat2))
% \end{verbatim}
%   Dat is the orginal data. Dat2 contains only the data, removing any
%   subjects containing errors. Vec1 is the data disregarding subject
%   level information.
%   \begin{enumerate}[a.]
%   \item Do the numbers 1-10 appear to be equally likely? Perform
%     the appropriate Chi-squared test.
%   \item Approximate an exact Chi-squared test by doing the following.
%     Simulate 1,000 random nultinomials under the null hypothesis with
%     the command
% \begin{verbatim}
% simdat <- t(rmultinom(1000, size = length(vec1), p = rep(.1, 10)))
% \end{verbatim}
%     Obtain the chi-squared statistics for each with the command
% \begin{verbatim}
% chsqStats <- apply(simdat, 1, function(x) chisq.test(x)$statistic)
% \end{verbatim}
%     Calculate the percentage of time that these statistics are greater
%     than the observed statistic. Explain how, provided the Monte Carlo
%     sample is large, this is a P-value.  
% \end{enumerate}
\item The following data show the results of caries surveys in five towns
and also the fluoride content of the drinking water.
\begin{center}
\begin{tabular}{lcccccc}
\hline
& Surrey & & & & West & \\
Area & and Essex & Slough & Harwick & Burnham & Meresa & Total \\
Fluoride p.p.m.\ & 0.15 & 0.9 & 2.0 & 3.5 & 5.8 & \vspace{+0.05in} \\ \hline
Number children with &&&&&& \\
with caries & 243 & 83 & 60 & 31 & 39 & 456 \vspace{+0.10in} \\
Number children with &&&&&& \\
caries free teeth & 16 & 36 & 32 & 31 & 12 & 127 \vspace{+0.10in} \\
Number examined & 259 & 119 & 92 & 62 & 51 & 583 \\ \hline
\end{tabular}
\end{center}
The data refer to samples of children aged 12-14 only.  
\begin{enumerate}[a.]
\item Estimate the odds ratio for the propobability of caries for lowest and highest (.15 to 5.8)
  categories of flouride. Interpret these results.
\item Estimate the relative risk for the propobability of caries for lowest and highest (.15 to 5.8)
  categories of flouride. Interpret these results.
\item Estimate the risk difference for the propobability of caries for lowest and highest (.15 to 5.8)
  categories of flouride. Interpret these results.
\item Test the equivalence for the probability of caries between the lowest and highest flouride concentrations (.15 to 5.8)
  give a P-value and interpret your results.
\end{enumerate}
\item A case-control study of esophageal cancer was performed.  Daily alcohol
consumption was ascertained $(80 +$ gm $=$ high, $0-79$ gm $=$ low). The data
was stratified by 3 age groups.
\begin{center}
\begin{tabular}{lllclllclll}
& \multicolumn{2}{c}{Alcohol} &&& \multicolumn{2}{c}{Alcohol} 
&& &\multicolumn{2}{c}{Alcohol} \\
& H & L & && H & L &&& H & L \\
case & 8 & 5 && case & 25 & 21& & case & 50 & 61 \\ 
control & 52 & 164 && control & 29 & 138& & control & 27 & 208 \vspace{+0.25in} \\
& \multicolumn{2}{c}{Age 35-44} &&& \multicolumn{2}{c}{Age 45-54}
&&& \multicolumn{2}{c}{Age 55-64}
\end{tabular} 
\end{center}
Give a confidence interval estimate of the odds ratio in the Oldest age group.
\item  In a study of aquaporins, 120 frog eggs were randomized, 60 to
  receive a protein treatment and 60 controls. If the treatment of
  the protein is effective, the frog eggs would implode. The resulting
  data was
    \begin{center}
      \begin{tabular}{cccc}
              & Imploded & Did not & Total \\
      Treated & 50        & 10       & 60     \\
      Control & 20        & 40       & 60     \\
      Totals  & 70        & 50       & 120    \\       
      \end{tabular}
    \end{center}
    State the appropriate hypotheses and report and interpret a
    P-value. 
\item Let $\hat p$ be the sample proportion from a binomial experiment
with $n$ trials.  Recall that the standard error of $\hat p$ is
$\sqrt{\hat p (1 - \hat p) / n}$. Define $f(\hat p) = \log\{\hat p /
(1 - \hat p)\}$ as the sample log odds.  Note, the following
  fact might be useful:
$$
f'(x) = \frac{1}{x(1 - x)}
$$
Use the delta method to create a confidence interval
  for the sample log odds.
\item Refer to the previous problem. Let $\hat p_1$ be the sample proportion from
  one binomial experiment with $n_1$ trials and $\hat p_2$ be the
  sample proportion from a second with $n_2$ trials. Define the log
  odds ratio to be $f(\hat p_1) - f(\hat p_2)$. Use your answer to
  part 2 to derive a confidence interval for the log odds ratio.
\item Two drugs, $A$ and $B$ , are being investigated in a randomized
  trial with the data are given below.  Investigators would like to
  know if the Drug A has a greater probability of side effects than
  drug B.
  \begin{center}
    \begin{tabular}{|c|c|c|c|} \hline
             & None & Side effects  & $N$ \\\hline 
      Drug A &  10  &   30    & 40 \\ 
      Drug B &  30  &   10    & 40 \\ \hline
    \end{tabular}
  \end{center}
  \begin{enumerate}[a.]
  \item State relevant null and alternative hypotheses and perform the relevant test.
  \item Estimate a confidence interval for the odds ratio, relative risk and risk difference. Interpret these reslts.
  \end{enumerate}
\item You are flipping a coin and would like to test if it is fair. You flip it 
10 times and get 8 heads. Specify relevant hypotheses and report and interpret an exact P-value.
\end{enumerate}


\end{document}
