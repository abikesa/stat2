\documentclass[12pt]{article}
\usepackage{geometry,amsmath,amssymb, graphicx, natbib, float, enumerate}
\geometry{margin=1in}
\renewcommand{\familydefault}{cmss}
\restylefloat{table}
\restylefloat{figure}

\newcommand{\code}[1]{\texttt{#1}}
\newcommand{\Var}{\mathrm{Var}}
\newcommand{\logit}{\mathrm{logit}}
\newcommand{\alone}{[{\bf ALONE}]~}

\begin{document}
\noindent
{\bf BST 140.651 \\ Problem Set 3} \\

\begin{enumerate}[Problem 1.]
\item Imagine that a person, say his name is Flip, has an oddly
  deformed coin and tries the following experiment.  Flip flips his
  coin $10$ times, $7$ of which are heads. You think maybe Flip's coin
  is biased towards having a greater probability of yielding a head than 50\%.
  \begin{enumerate}[a.]
  \item What is the maximum likelihood estimate of $p$, the true probability of
    heads associated with this coin?
  \item Plot the likelihood associated with this experiment. Renormalize the
    likelihood so that its maximum is one. Does the likelihood suggest that
    the coin is fair?
  \item Whats the probability of seeing $7$ or more heads out of ten
    coin flips if the coin was fair? Does this probability suggest
    that the coin is fair? Note this number is called a P-value.
  \item Suppose that Flip told you that he did not fix the number of
    trials at $10$. Instead, he told you that he had flipped the coin
    until he obtained $3$ tails and it happened to take $10$ trials to
    do so. Therefore, the number $10$ was random while the number
    three $3$ fixed.  The probability mass function for the number of
    trials, say $y$, to obtain $3$ tails (called the negative binomial
    distribution) is
    $$
    \left(\begin{array}{c}y-1 \\ 2\end{array}\right)(1-p)^3p^{y - 3}
    $$
    for $y=3,4,5,6,\ldots$. What is the maximum likelihood estimate of
    $p$ now that we've changed the underlying mass function?
  \item Plot the likelihood under this new mass function. Renormalize
    the likelihood so that its maximum is one. Does the likelihood
    suggest that the coin is fair?
  \item Calculate the probability of requiring $10$ or more flips to
    obtain $3$ tails if the coin was fair. (Notice that this is the
    same as the probability of obtaining $7$ or more heads to obtain
    $3$ tails.) This is the Pvalue under
    the new mass function.
    \ \\ \ \\
    (Aside) This problem highlights a distinction between the
    likelihood and the P-value. The likelihood and the MLE are the
    same regardless of the experiment. That is to say, the likelihood
    only seems to care that you saw $10$ coin flips, $7$ of which were
    heads. Flip's intention about when he stopped flipping the coin,
    either at $10$ fixed trials or until he obtained $3$ tails, are
    irrelevant as far as the likelihood is concerned. The P-value, in
    comparison, does depend on Flip's intentions.
  \end{enumerate}
\item  Suppose a researcher is studying the number of sexual
  acts with an infected person until an uninfected person contracts an
  sexually transmitted disease. She assumes that each encounter is an
  independent Bernoulli trial with probability $p$ that the subject
  becomes infected. This leads to the so-called geometric distribution
  $P(\mbox{Person is infected on contact } x) = p(1 - p)^{x-1}$ for
  $x=1,\ldots$.
  \begin{enumerate}[a.]
  \item Suppose that one subject's number of encounters until
    infection is recorded, say $x$. Symbolically derive the ML
    estimate of $p$.
  \item Suppose that the subjects value was $2$. Plot and interpret
    the likelihood for $p$.
  \item Suppose that is often assumed that the probability of
    transmission, $p$, is $.01$. The researcher thinks that it is
    perhaps strange to have a subject get infected after only $2$
    encounters if the probability of transmission is really on
    $1\%$. According to the geometric mass function, what is the
    probability of a person getting infected in $2$ or fewer
    encounters if $p$ truly is $.01$?
  \item Suppose that she follows $n$ subjects and records the number
    of sexual encounters until infection (assume all subjects became
    infected) $x_1,\ldots,x_n$. Symbolically derive the ML
    estimate of $p$.
  \item Suppose that she records values $x_1 = 3$, $x_2 = 5$, $x_3 = 2$.
    Plot and interpret the likelihood for $p$.
  \end{enumerate}
\item  In a study of aquaporins $6$ frog eggs received a protein
  treatment. If the treatment of the protein is effective, the frog
  eggs would implode. The experiment resulted in $5$ frog eggs
  imploding.  Historically, ten percent of eggs implode without the
  treatment. Assuming that the results for each egg are independent
  and identically distributed:
  \begin{enumerate}[a.]
  \item What's the probability of getting $5$ or more eggs imploding
    in this experiment if the true probability of implosion is $10\%$?
    Interpret this number.
  \item What is the maximum likelihood estimate for the probability of
    implosion?
  \item Plot and interpret the likelihood for the probability of implosion.
  \end{enumerate}

\item (Adapted from Rosner page 135) Suppose that the diastolic blood
  pressures of $35-44$ year old men are normally distributed with mean
  $80$ ($mm$ $Hg$) and variance $144$. For the same population, the
  systolic blood pressures are also normally distributed and have a
  mean of $120$ and variance $121$.
  \begin{enumerate}[a.]
  \item What is the probability that a randomly selected person from
    this population has a DBP less than 90?
  \item What DBP represents the $90^{th}$, $95^{th}$ and $97.5^{th}$
    percentiles of this distribution?
  \item What's the probability of a random person from this population
    having a SBP $1$, $2$ or $3$ standard deviations above $120$?
    What's the corresponding probabilities for having DBPs $1$, $2$ or
    $3$ standard deviations above $80$?
  \item Suppose that $10$ people are sampled from this population.  What's
    the probability that 50\% (5) of them have a SBP larger than $140$?
  \item Suppose that $1,000$ people are sampled from this population.  What's
    the probability that 50\% (500) of them have a SBP larger than $140$?
  \item If a person's SBP and DBP are independent, what's the
    probability that a person has a SBP larger than $140$ and a DBP
    greater than $90$? Is independence a good assumption?
  \item Suppose that an average of $200$ people are drawn from
    this population. What's the probability that this average
    is smaller than $81.3$?
  \end{enumerate}
\item  Suppose that IQs in a particular population are normally
  distributed with a mean of $110$ and a standard deviation of $10$.
  \begin{enumerate}[a.]
  \item What's the probability that a randomly selected person from
    this population has an IQ between $95$ and $115$?
  \item What's the $65^{th}$ percentile from this distribution?
  \item Suppose that $5$ people are sampled from this distribution.
    What's the probability $4$ (80\%) or more have IQs above $130$?
  \item Suppose that $500$ people are sampled from this distribution.
    What's the probability $400$ (80\%) or more have IQs above $130$?
  \item Consider the average of $100$ people drawn from this
    distribution. What's the probability that this mean is larger
    than $112.5$?
  \end{enumerate}
\item  Suppose that $400$ observations are drawn at random
  from a distribution with mean $0$ and standard deviation $40$.
  \begin{enumerate}[a.]
  \item What's the approximate probability of getting a sample mean larger
    than $3.5$?
  \item Was normality of the underlying distribution required for this
    calculation?
  \end{enumerate}
\item Recall that R's function \texttt{runif} generates (by default)
  random uniform variables that have means $1/2$ and variance $1/12$.
  \begin{enumerate}[a.]
  \item Sample $1,000$ observations from this distribution. Take the
    sample mean and sample variance. What numbers should these
    estimate and why?
  \item Retain the same $1,000$ observations from part a. Plot the
    sequential sample means by observation number. Hint. If $x$ is a
    vector containing the simulated uniforms, then the code \texttt{y
      <- cumsum(x) / (1 : length(x))} will create a vector of the
    sequential sample means. Explain the resulting plot.
  \item Plot a histogram of the $1,000$ numbers. Does it look like a
    uniform density?
  \item Now sample $1,000$ {\em sample means} from this distribution,
    each comprised of $100$ observations. What numbers should the
    average and variance of these $1,000$ numbers be equal to and why?
    Hint. The command
    \begin{quote}
      \texttt{x <- matrix(runif(1000 * 100), nrow = 1000)}
    \end{quote}
    creates a matrix of size $1,000\times 100$ filled with random
    uniforms. The command \texttt{y<-apply(x,1,mean)} takes the sample
    mean of each row.
  \item Plot a histogram of the $1,000$ sample means appropriately
    normalized. What does it look like and why?
  \item Now sample $1,000$ {\em sample variances} from this distribution,
    each comprised of $100$ observations. Take the average of these
    $1,000$ variances. What property does this illustrate and why?
  \end{enumerate}
\item Note that R's function \texttt{rexp} generates random
  exponential variables. The exponential distribution with rate $1$
  (the default) has a theoretical mean of $1$ and variance of $1$.
  \begin{enumerate}[a.]
  \item Sample $1,000$ observations from this distribution. Take the
    sample mean and sample variance. What numbers should these
    estimate and why?
  \item Retain the same $1,000$ observations from part a. Plot the
    sequential sample means by observation number. Explain the
    resulting plot.
  \item Plot a histogram of the $1,000$ numbers. Does it look like a
    exponential density?
  \item Now sample $1,000$ {\em sample means} from this distribution,
    each comprised of $100$ observations. What numbers should the
    average and variance of these $1,000$ numbers be equal to and why?
 \item Plot a histogram of the $1,000$ sample means appropriately
    normalized. What does it look like and why?
  \item Now sample $1,000$ {\em sample variances} from this distribution,
    each comprised of $100$ observations. Take the average of these
    $1,000$ variances. What property does this illustrate and why?
  \end{enumerate}
\item  Consider the distribution of a fair coin flip (i.e. a random variable
  that takes the values $0$ and $1$ with probability $1/2$ each.)
  \begin{enumerate}[a.]
  \item Sample $1,000$ observations from this distribution. Take the
    sample mean and sample variance. What numbers should these
    estimate and why?
  \item Retain the same $1,000$ observations from part a. Plot the
    sequential sample means by observation number. Explain the
    resulting plot.
  \item Plot a histogram of the $1,000$ numbers. Does it look like it places
    equal probability on $0$ and $1$?
  \item Now sample $1,000$ {\em sample means} from this distribution,
    each comprised of $100$ observations. What numbers should the
    average and variance of these $1,000$ numbers be equal to and why?
 \item Plot a histogram of the $1,000$ sample means appropriately
    normalized. What does it look like and why?
  \item Now sample $1,000$ {\em sample variances} from this distribution,
    each comprised of $100$ observations. Take the average of these
    $1,000$ variances. What property does this illustrate and why?
  \end{enumerate}
\item Consider a density for the proportion of a person's body that is
  covered in freckles, $X$, given by $f(x) = cx$ for $0 \leq x \leq 1$
  and some constant $c$.
  \begin{enumerate}[a.]
  \item What value of $c$ makes this function a valid density?
  \item What is the mean and variance of this density?
  \item You simulated $100,000$ sample means, each comprised of $100$
    draws from this density. You then took the variance of those
    $100,000$ numbers. Approximately what number did you obtain?
    (Explain.)
  \end{enumerate}
\item Suppose that DBPs drawn from a certain population are normally
    distributed with a mean of $90$ $mmHg$ and standard deviation of $5$
    $mmHg$. Suppose that $1,000$ people are drawn from this population.
  \begin{enumerate}[a.]
  \item If you had to guess the number of people people in having DBPs
    less than $80$ $mmHg$ what would you guess?
  \item  You draw
    $25$ people from this population. What's the probability tha the
    sample average is larger than $92$ $mmHg$?
  \item  You select $5$
    people from this population. What's the probability that $4$ or
    more of them have a DBP larger than $100$ $mmHg$?
  \end{enumerate}
\item You need to calculate the probability that a {\em standard normal} is
  larger than $2.20$, but have nothing available other than a regular coin.
  Describe how you could estimate this probability using only your coin. (Do not
  actually carry out the experiment, just describe how you would do it.)
\item Let $X_1$, $X_2$ be independent, identically distributed coin
  flips (taking values $0$ = failure or $1$ = success) having success
  probability $\pi$. Give and interpret the likelihood ratio comparing
  the hypothesis that $\pi = .5$ (the coin is fair) versus $\pi = 1$ (the
  coin always gives successes) when both coin flips result in
  successes.
\item The density for the population of increases in wages for
assistant professors being promoted to associates (1 = no increase, 2 = salary has
doubled) is uniform on the range from 1 to 2.
\begin{enumerate}[a.]
\item What's the mean and variance of this density?
\item Suppose that the sample variance of $10$ observations from this
  density was sampled say $10,000$ times. What number would we expect
  the average value from these $10,000$ variances to be near? (Explain
  your answer briefly.)
\end{enumerate}
\item Suppose that the US intelligence quotients (IQs) are normally
  distributed with mean $100$ and standard deviation $16$.
  \begin{enumerate}[a.]
  \item  What IQ
    score represents the $5^{th}$ percentile? (Explain your calculation.)
  \item Consider the previous question. Note that $116$ is the
    $84^{th}$ percentile from this distribution. Suppose now that
    $1,000$ subjects are drawn at random from this population. Use the
    central limit theorem to write the probability that less than
    $82\%$ of the sample has an IQ below $116$ as a standard normal
    probability. Note, you do not need to solve for the final number. (Show your work.)
  \item Consider the previous two questions. Suppose now that a sample
    of $100$ subjects are drawn from a {\em new} population and that
    $60$ of the sampled subjects had an IQs below $116$. Give a $95\%$
    confidence interval estimate of the true probability of drawing a
    subject from this population with an IQ below $116$. Does this proportion
    appear to be different than the $84\%$ for the population from questions 1 and 2?
  \end{enumerate}

  \item Let $X$ be binomial with success probability $p_1$ and $n_1$
    trials and $Y$ be an independent binomial with success probability
    $p_2$ and $n_2$ trials.  Let $\hat p_1 = X / n_1$ and $\hat p_2 =
    Y / n_2$ be the associated sample proportions. What would be an
    estimate for the standard error for $\hat p_1 - \hat p_2$? To have
    consistent notation with the next problem, label
    this value $\hat{SE}_{\hat p_1 - \hat p_2}$.

\item You are in desperate need to simulate standard normal random
  variables yet do not have a computer available. You do, however, have
  ten standard six sided dice. Knowing that the mean of a single die roll
  is $3.5$ and the standard deviation is $1.71$, describe how you
  could use the dice to approximately simulate standard normal random variables. (Be precise.)
\item In a sample of $40$ United States men contained $25\%$
  smokers. Let $p$ be the true prevalence of smoking amongst males in
  the United States. Write out and draw and interpret the likelihood
  for $p$. Is $p=.35$ or $p=.15$ better supported given the data (why,
  and by how much)? What value of $p$ is best supported (just give the
  number, do not derive)?
\item Consider three sample variances, $S_1^2$, $S_2^2$ and
  $S_3^2$. Suppose that the sample variances are comprised of $n_1$,
  $n_2$ and $n_3$ iid draws from normal populations $N(\mu_1,
  \sigma^2)$, $N(\mu_2, \sigma^2)$ and $N(\mu_3, \sigma^2)$, respectively. Argue
  that $$\frac{(n_1 - 1)S_1^2 + (n_2 - 1)S_2^2 + (n_3 - 1) S_3^2}{n_1
    + n_2 + n_3 - 3}$$ is an unbiased estimate of $\sigma^2$.
\item You need to calculate the probability that a normally
  distributed random variable is less than $1.25$ standard deviations
  below the mean. However, you only have an oddly shaped coin with a
  known probability of heads of $.6$. Describe how you could estimate
  this probability using this coin. (Do not actually carry out the
  experiment, just describe how you would do it.)
\item The next three questions (A., B., C.) deal with the following
  setting. Forced expiratory volume, $FEV_1$, is a measure of lung
  function that is often expressed as a proportion of lung capacity
  called forced vital capacity, FVC.  Suppose that the population
  distribution of $FEV_1/FVC$ of asthmatics adults in the US has mean
  of $.55$ and standard deviation of $.10$.
\begin{enumerate}[A.]
\item Suppose a random sample of $100$ people are drawn from this
  population. What is the probability that their average $FEV_1/FVC$ is
  larger than $.565$?
\item Suppose the population of non-asthmatics adults in the US have
  a mean $FEV_1/FVC$ of $.8$ and a standard deviation of $.05$.
  You sample $100$ people from the asthmatic population and
  $100$ people from the non-asthmatic population and take the
  difference in sample means. You repeat this process $10,000$
  times to obtain $10,000$ differences in sample means. What
  would you guess the mean and standard deviation of these
  $10,000$ numbers would be?
\item Moderate or severe lung dysfunction is defined as $FEV_1/FVC
  \leq .40$. A colleague tells you that $60\%$ of asthmatics
  in the US have moderate or severe lung dysfunction. To verify this,
  you take a random sample of $5$ subjects, only one of which has
  moderate or severe lung dysfunction. What is the probability of
  obtaining only one or fewer if your friend's assertion is
  correct? What does your result suggest about their
  assertion?
\end{enumerate}
\item Consider a sample of $n$ iid draws from an exponential density
$$
\frac{1}{\beta}\exp(-x / \beta) ~~\mbox{for}~~ \beta > 0.
$$
\begin{enumerate}[A.]
\item Derive the maximum likelihood estimate for $\beta$.
\item Suppose that in your experiment, you obtained five
observations
\begin{verbatim}
 1.590 0.109 0.155 0.281 0.453
\end{verbatim}
plot the likelihood for $\beta$. Put in reference lines at 1/8 and 1/16.
\end{enumerate}
\item Often infection rates per time at risk are modelled as Poisson
random variables. Let $X$ be the number of infections and let $t$ be the
person days at risk. Consider the Poisson mass function
$
(t\lambda)^x \exp(-t\lambda) / x!
$
The parameter $\lambda$ is called the population incident rate.
\begin{enumerate}[A.]
\item Derive the ML estimate for $\lambda$.
\item Suppose that 5 infections are recorded per 1000 person-days at risk. Plot the likelihood.
\item Suppose that five independent hospitals are monitored and that the infection rate ($\lambda$) is assumed to be the same at all five. Let $X_i$, $t_i$ be the count of the number of infections
and person days at risk for hospital $i$. Derive the ML estimate of $\lambda$.
\end{enumerate}
\item Consider $n$ iid draws from a gamma density where $\alpha$ is known
$$
\frac{1}{\Gamma(\alpha)\beta^\alpha} x^{\alpha -1} \exp(-x/\beta) ~~~\mbox{for}~~~\beta > 0, x>0, \alpha > 0.
$$
\begin{enumerate}[A.]
\item Derive the ML estimate of $\beta$.
\item Suppose that $n=5$ observations were obtained: 0.015, 0.962, 0.613, 0.061, 0.617. Draw
    a likelihood plot for $\beta$ (still assume that $\alpha = 1$).
\end{enumerate}
\item Let $Y_1,\ldots, Y_N$ be iid random variabels from a Lognormal distribution with parameters
    $\mu$ and $\sigma^2$. Note $Y \sim \mbox{Lognormal}(\mu,\sigma^2)$ if and only if
    $\log Y \sim N(\mu,\sigma^2)$. The log-normal density is given by
    $$
    (2\pi \sigma^2)^{-1/2} \exp[-\{\log(y) - \mu\}^2 / 2\sigma^2] / y ~~\mbox{for}~~ y > 0
    $$
    \begin{enumerate}[A.]
    \item Show that the ML estimate of $\mu$ is $\hat \mu = \frac{1}{N} \sum_{i=1}^N \log(Y_i)$. (The
    mean of the log of the observations. This is called the ``geometric mean''.)
    \item Show that the ML estimate of $\sigma^2$ is then the biased variance estimate based
    on the log observation
    $$
    \frac{1}{N}\sum_{i=1}^N (\log(y_i) - \hat \mu)^2
    $$
    \end{enumerate}
  \end{enumerate}
\end{document}
